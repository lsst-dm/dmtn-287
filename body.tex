\section{Introduction}
In 2019 the funding of Vera C.\ Rubin Observatory \cite{2019ApJ...873..111I} operations changed with the Department of Energy (DOE) increasing its contribution to 50\%, with the bulk of that funding the US Data Facility (USDF) at a site to be determined.
This led to changes in how and where we would operate Rubin Data Management.
We used the opportunity and uncertainty to propose the Interim Data Facility (IDF), a cloud-based solution, thus alleviating the immediate need to know the location of the USDF.
The IDF has been very successful and supported three data previews with simulated data.\cite{2021arXiv211115030O}
When SLAC National Accelerator Laboratory was selected as the USDF we maintained the interim solution to overlap the startup with the USDF; however, as we discussed the architecture a hybrid solution emerged.
Keeping all science users on the cloud has certain security and scalability advantages while keeping the bulk of the data at SLAC has some cost advantages.
DOE has committed funds for three years of the US cloud-based Data Access Center (DAC) on Google, which should bring us to 2027.
The interim cloud will transition to become the US DAC.
For the first two data previews all data was on Google; the third data preview had the database at SLAC and users on Google.
The intention is to have most data at SLAC with the users accessing databases using IVOA protocols and images using the client/server Butler.\cite{2024SPIE13101.129Jtmp}
Thus the users do not have SLAC accounts and do not require approval through more detailed institutional processes.
The system is built on Terraform and Kubernetes deployed with ArgoCD using our own configuration system, Phalanx.\footnote{\url{https://phalanx.lsst.io}}

\section{System Requirements} \label{sec:requirements}

Rubin Observatory has a rigorous approach to requirements management \cite{2016SPIE.9911E..0DS}.
Most relevant requirements for the USDF are in the Data Management Subsystem Requirements(DMSR) document \cite{LSE-61}.
Having to switch data facilities encouraged us to pull system requirements which affect the data facility into one document
which was used to scope the SLAC operations \cite{rtn-080}.
In this section we will not enumerate requirements, however the tables of requirements broadly matching this section may be found in \cite{rtn-080}.

In summary the USDF is responsible for significant functionality in several areas as outlined below.


\subsection{Networking } \label{sec:networking}

USDF must arrange 100Gbit/s, path redundant, network capacity to the Energy Sciences Network (ESNet) to connect to the Rubin Observatory facility in Chile.
USDF must ensure their contribution to network latency is maximum 3s.
Enough bandwidth must also be available to exchange files with France and the UK for annual Data Release processing.

\subsection{Prompt processing} \label{sec:prompproc}
The USDF will need to run the Prompt Processing framework in near-real-time in order to execute the Alert Production payload that generates prompt data products and alerts corresponding to changes in the sky.
The processing is to be completed and alerts are to be distributed within two minutes of the end of readout of an image from the LSST Camera.
Quality control metrics for the images also need to be generated and made available to staff.

Prompt products, including both images and catalogs, and alerts are stored for retrieval by science users and staff.

\subsection{Batch System} \label{sec:offlineprod}
Every year, the accumulated images taken to date will be reprocessed.
This extensive and complex Data Release Production runs in batch mode across the US, French, and UK Data Facilities.
The USDF is responsible for providing infrastructure for executing 35% of the Data Release, coordinating the campaign to generate the annual Data Release, ensuring the quality of the data products, archiving a copy of 100% of those products, and making them available to science users through the US DAC.

Certain products will also be distributed to Independent Data Access Centers and Science Collaborations.

The batch system and associated interactive nodes are also used extensively by staff for development of future versions of the LSST Science Pipelines code.

\subsection{Data transfer and preservation} \label{req:dbb}
The USDF is responsible for operating the systems that track all raw data and released data products, including managing their movement, backup, and lifetime.
These systems must be fault-tolerant and able to catch up after failures.

\subsection{US Data Access Center}
The USDF hosts the US DAC.
To science users, the DAC is primarily an installation of the Rubin Science Platform (RSP).
This software includes the Portal Aspect, a web-based application for browsing, querying, and investigating the data products; the Notebook Aspect, which allows interactive, customized programs to retrieve and manipulate the data products; and the API Aspect, which provides community-standard protocols for automated retrieval of data.

The RSP is deployed using Helm and ArgoCD as a suite of services on top of Kubernetes.

In the hybrid model, most of this will now be hosted in the cloud, with the underlying data, both prompt products and annual Data Release products, fed from the USDF at SLAC.
But an RSP for staff will be deployed on top of a Kubernetes cluster provided by the USDF.

There are requirements for science users to have some access to batch-type processing for larger-scale, non-interactive computations on the data.

This system is to be sized for around 10K users with perhaps 1K simultaneously accessing at any given time.

There is further functionality specified for the DAC such as precovery, product regeneration and special program support.
User generated products must be stored and potentially shared; catalog uploads will also be allowed.
Access is not only to the current Release but also one previous Release of the data.

\section{General cloud benefits and drawbacks} \label{sec:tradeoffs}

One of the major advantages of this approach is the separation of security concerns, users  SLAC require extra checks which take some time because it is a DOE facility.
SLAC may have difficulty processing accounts  for  our many thousands of users since their process is somewhat lengthy.
By putting all the science users on Google we can streamline access using InCommon \footnote{\url{https://incommon.org/}} which will allow us to identify most US academic users.

We have found Google during the data preview very performant and reliable - we may give up some of this hosting the image data at SLAC.
In our first SLAC year we have had some data loss and some scaling issues.

A major plus for on prem in this instance is the backing of DOE - this is what they want and there is commitment to making it work.


\section{Architecture on Google cloud} \label{sec:google}

Here we list the cloud components.

We reported on the Interim Data Facility (IDF) already in \cite{DMTN-209}.


\begin{figure}
\begin{centering}
\includegraphics[width=0.9\textwidth]{RSP.png}
	\caption{ Users hosted on Google will typically use the Rubin Science Platform (RSP) depicted here.  \label{fig:goglearch}}
\end{centering}
\end{figure}

Here we enumerate on-prem components.
\section {US Data Facility Architecture} \label{sec:usdfarch}


\begin{figure}
\begin{centering}
\includegraphics[width=0.9\textwidth]{hybrid}
	\caption{ Hybrid model: Data at SLAC but users on the Cloud.  \label{fig:usdfarch}}
\end{centering}
\end{figure}

The scope for the USDF on-prem includes data production services:
prompt processing, serving alerts to the community and annual Data
Release Processing. The USDF acts as the archive for all data, and
provides the Qserv object catalog as well as access to image data, be
it cutouts or full images. It will provide batch cycles for cloud-based science users.

It will also act as a home for developers and staff (and
commissioners) to ensure data quality



\subsection{hardware (storage, compute, networking)}

The USDF is hosted by the S3DF which is itself hosted in SRCF. SRCF
acommodates projects from SLAC and Stanford, while the S3DF is the
focal system for SLAC projects. The USDF lives in a shared cluster and
benefits from economies of scale and standardization across S3DF
projects. It is also exposed to potentially disruptive activities by
other projects.

In order to support hundreds of PBs of storage, S3DF adopted the weka
filesystem for high throughput. Weka is based on a tiered system with
SSD backed by spinning disk. It presents a POSIX interface while the
backend is a CEPH object store. This system forms the basis of the
data archive. A tape robot provides storage for seldom-read data and
acts as a backup tier.

Batch processing is done on a slurm cluster, currently primarily milan AMD
processors with 128 cores and 512 GB RAM per node.

Data is transported to the USDF fron the summit over a combined
leased-line, ESNet supported network with routing optimized via an
overlay. The leased line terminates in Atlanta, where ESNet takes
over. Traffic to the two other Data Facilities is also provided by
ESNet, connecting to the GEANT and Renater systems in Europe.

\subsection{batch processing/data management using PanDA, condor,
  slurm; Rucio/butler}

The USDF supports batch processing for a number of purposes: annual
multi-site data releases; pipelines teams testing for algorithms
performance; processing by individual developers for their algorithm
development; data quality checking and validation.

Multisite processing makes use of the PanDA workflow system, developed
by ATLAS for the LHC. It has well defined mechanism for routing work
from a central server to multiple remote locations. ATLAS has
demonstrated submitting millions of jobs per day to hundreds of
sites. A difference between typical astronomy and HEP workflows is the
number of and duration of processes: astronomy tends to many more much
shorter jobs than HEP. Significant effort was required with the PanDA
team to cluster up short jobs to avoid prohibitive startup costs.

PanDA is a heavyweight solution to processing; local processing for
the pipelines teams and developers is done using HTCondor.

Data management and movement is also orchestrated by LHC tools: Rucio
for data managment and FTS3 for movement. These tools also routinely
handle large numbers of files and transfers, but the astronomy:HEP
difference persists here as well, with astronomy generating many more,
much smaller files than HEP. This will make the Rubin Rucio database
bigger than ATLAS's and will required some growth planning.

The large number of small files will also be a challenge for network
transfer. We are investigating zipping up large numbers of files both
for better transfer as well as easier storage on tape.

\subsection{Non-user-facing services and why they are on-prem}

Currently the primary reasons for putting services on-prem are a low
latency requirement for prompt processing, and the still-unfavorable
comparison of storage prices between on-prem and cloud. To a lesser
degree, those comparisons also apply to CPU.

This means that Prompt Processing and Alerts production, with their
2-minute latency requirement are hosted on-prem. Additionally, there
are security requirements on data arriving at the USDF, including
physical measures implemented on the racks themselves.

The large data volumes associated with the storage archive and Qserv
database hosted at the USDF implies that external access to them must be
provided by services.

Kubernetes is used to manage almost all our services, making use of
ARGO-CD as well as our custom Phalanx system (see \S \ref{sec:deploy}). Native kubernetes tools
are used to manage standard services, such as postgres databases,
making administration, backups etc scalable. Rucio and PanDA are
managed by kubernetes to take advantage of these features.

The Prompt Processing and Alerts Production are also implemented in
kubernetes to allow elastic instanciation, configuration and teardown of
pods responding to notifications from the summit in advance of the
next visit.

Three large database systems are mininally using kubernetes, as they
are either commercial or custom services with no native kubernetes
support. These are the Engineering and Facilities Database (EFD),
Qserv and Cassandra systems, with Qserv the custom system. EFD is
implemented with an InfluxDB Enterprise HA cluster.

\section{Continuous Deployment across the Rubin facilities} \label{sec:deploy}

Early on we adopted Kubernetes which provided a service architectures that is well isolated from the underlying infrastructure.
This approach has already paid off massive dividends.
For example when funding lines suddenly shifted we were able to painlessly transition from an on-premises facility to an Interim Data Facility on Google Cloud.
Furthermore the Rubin Science Platform (RSP) became a generic data services platform that is currently deployed on eight distinct (and distinctly managed) infrastructures (on-prem and cloud).
Finally this has allowed us to leverage the Cloud for services like the RSP which benefit from its advantages such as elasticity, scalability, isolation.


\subsection{Division of responsibilities}
We have had ways to abstract system services from our developed services for some time, containers and java effectively allow one to run \emph{anywhere}.
These work well enough for single user single processor deployments.
When you need to scale up another level of orchestration is needed, kubernetes fits neatly between our infrastructure and our developed applications providing a powerful container orchestration and resource management system.

Our agreement with each facility is to provide a Kubernetes deployment platform upon which we may deploy our services.
It feels like the first time that this actually works well - we have to learned to make our applications portable of course but also the technology is so much better than any precursors.

Each thus provides a kubernetes environment with disk, network and compute resources upon which we spin up our services.
A vault for secrets is also needed.
From the service perspective each facility looks similar.

\subsection{Phalanx, Helm and ArgoCD}

Helm\footnote{https://helm.sh/} is a standard approach to tell Kubernetes about software.
For our purposes this is most usually a JSON file describing which github repo our code is in where the container is and how to start it.

Of course there are parts of this configuration which are site specific and parts which are generic.
For example the database URL for an application will be different in USDF and Summit, the \emph{value} of the database URL is different are each site, the URL variable is the same so the application does not change only the configuration.
Phalanx\footnote{phalanx.lsst.io} is our repository of configuration files for all of our deployments.
Within phalanx each application has a directory with a helm configuration - it also has a values file for each environment to specify the specific values for that environment.
Each environment also has a Vault configuration so that secrets such as database password may also be specified by an environment specific identifier.
Finally Phalanx has a list of deployed applications on each site.

Actually deployment is manged by ArgoCD\footnote{https://argo-cd.readthedocs.io/en/stable/}.
As the name implies ArgoCD provided continuous delivery for Kubernetes.
It interprets the configuration files stored in Phalanx to deploy containers on the kubernetes system associated with each environment.
Some of the current environments are listed in \autoref{tab:envs}, typically each environment also has a \emph{dev} and \emph{integration} version.
This is a powerful system which can track the main or any given branch or tag of a github repository to keep the application in sync.
\begin{longtable}{l|l}
\caption {Phalanx environments - typically we have dev, int and production for each.}\label{tab:envs}\\
\hline
\textbf{Name} & \textbf{Environment endpoint}  \\\hline
base & La Serena \\\hline
ccin2p3 & French Data Facility \\\hline
idfprod & Production RSP in GCP \\\hline
minikube & GitHub Actions CI \\\hline
roe & UK Data Facility \\\hline
roundtable-prod & SQuaRE services \\\hline
summit & Rubin Summit \\\hline
tucson-teststand & T\&S/SITCom \\\hline
usdfprod &  Production RSP at USDF \\\hline
usdfprod-prompt-processing & Prod for USDF Prompt Processing \\\hline
\end{longtable}


\subsection{Continuous Integration}
We rely on Github actions to build containers for each branch or tag of a given application


\section{Open Issues} \label{sec:open}

We have not yet tested the client server butler between SLAC and Google.
We have checked the bandwidth and latency is acceptable but it remains to be seen how it will work under load.

Butler implements a client cache  so a scientists notebook or script  repeatedly accessing the same image will be well handled.
But for the cloud connections we consider a general cache for images,  this would only work if many scientists use the same image.
We do not know the image access patterns that will occur in operations and hence are not sure if this will work or perhaps the client cache is adequate.


We need to investigate how to get a  consolidated view of key metrics and monitoring across the cloud and on-premise.
Such a view would help immensely in tracking down issues which may appear in one part while the cause is elsewhere.

For user batch we have specified users would have to have SLAC accounts and log in to SLAC to submit jobs \cite{DMTN-223}.
It would be user friendly to allow submission of batch jobs directly from the cloud hosted science platform which execute on SLAC.
In principle we can do this but there are issues ranging from time allocation at SLAC and viewing of results which are not yet resolved.


~



\section{Conclusion} \label{sec:conc}
text here

