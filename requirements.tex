\section{System Requirements} \label{sec:requirements}

Rubin Observatory has a rigorous approach to requirements management \cite{2016SPIE.9911E..0DS}.
Most relevant requirements for the USDF are in the Data Management Subsystem Requirements(DMSR) document \cite{LSE-61}.
Having to switch data facilities encouraged us to pull system requirements which affect the data facility into one document
which was used to scope the SLAC operations \cite{rtn-080}.
In this section we will not enumerate requirements however the tables of requirements broadly matching this section may be found in \cite{rtn-080}.

In summary the USDF is responsible for significant functionality in several areas as outlined below.


\subsection{Networking } \label{sec:networking}

USDF must arrange 100Gbs, path redundant, network capacity to Energy Sciences Network (ESNet) to connect to the Rubin Observatory facility in Chile.
USDF must ensure their contribution to network latency is maximum 3s.
Enough bandwidth must also be available to send files to France and UK.

\subsection{Prompt processing} \label{sec:prompproc}
The USDF will need to run certain software in near real time namely:
\begin{itemize}
\item Alert Distribution within two minutes
\item Prompt Processing Ingest, long term storage of images
\item Prompt Quality Control, generation of metrics within the two minute prompt processing period
\item Prompt Processing, processing each image in under two minutes to produce prompt products and alerts
\item Offline Quality Control, generation of more detailed and cross visit metrics the next day
\end{itemize}
This is referred to as the prompt enclave in \cite{DMTN-104}.
This implies storing camera images and engineering data and making it available to team members within 24 hours.
It also implies underlying database for the Prompt and Alert product databases.

\subsection{Batch System} \label{sec:offlineprod}

The USDF will need to run certain software in batch mode:
\begin{itemize}
\item Batch Production
\item Offline Quality Control
\item Bulk Distribution
\end{itemize}
This is referred to as the offline production  enclave in \cite{DMTN-104}.
This implies availability of pipelines, data and infrastructure to run them on.
The data here includes calibration and WaveFront sensor data.
IN addition to the DM users Education and Public Outreach (EPO) lay some requirements in this area also - though what exactly remains TBD.


\subsection{US Data Access Center}
The USDF will need to host the US DAC comprising the elements below :
\begin{itemize}
\item LSP Nublado
\item LSP Portal
\item WebDAV API
\item SIA API
\item SODA API
\item TAP API
\item LSP Database
\end{itemize}
This is referred to as the DAC US Enclave in \cite{DMTN-104}.
Most of this will now be hosted on google.

Some implications are that nightly data be available 80 hours after shutter close, keeping of historical alerts and allowing some user access to batch processing.
This system is to be sized for around 10K users with perhaps 1K simultaneously accessing at any given time.
There is further functionality specified for the DAC such as precovery, product regeneration and special program support.
User generated products must be stored and potentially shared, catalog uploads will also be allowed.
Access is not only to the current Release but also one previous Release of the data.

\subsection{Data transfer and preservation} \label{req:dbb}

The USDF needs to look after Rubin data in terms of :
\begin{itemize}
	\item Ingest/ Metadata Management
	\item Lifetime Management
	\item Transport/ Replication/ Backup
	\item Storage
\end{itemize}

This is referred to as DataBackbone services  in \cite{DMTN-104}.
This also implies building in fault tolerance, sizing for catch-up processing and transfer, backup and recovery.

\subsection{Science Platform and Kubernetes} \label{sec:rsp}
The USDF was also required to be ready to deploy the Rubin Science Platform.
This is a suite of services deployed on top  of Kubernetes using Helm and ArgoCD.
Though we use puppet in Chile - USDF use ansibile and chef.

The USDF must provide and up to date Managed Kubernetes, including all necessary administrative access to create/destroy/administer clusters and debug pod and storage problems. This also includes:

\begin{enumerate}
\item self serve tools for machine and cluster management. e.g. K8S admin.
\item  ability for Rubin engineers to solely or jointly manage ingress services to the Kubernetes cluster(s)
\item ability for Rubin services to utilize Kubernetes Dynamic Volume Provisioning.
\item allow Rubin services to control UID/GID of users in the POSIX filesystem
\item  allow select services pods (not users) to access storage with escalated privileges
\end{enumerate}

