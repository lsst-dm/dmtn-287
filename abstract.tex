
\begin{abstract}
Cloud computing offers unparalleled flexibility, a constantly increasing set of “Infrastructure As A Service’’ capabilities, resource elasticity and security isolation. One of the most significant barriers in astronomy to wholesale adoption of cloud infrastructures is the cost for hot storage of large datasets - particularly for Rubin, a Big Data project sized at 0.5 Exabytes (500 Petabytes) over the duration of its 10-year mission. We are planning to reconcile this with a “hybrid” model where user-facing services are deployed on Google Cloud with the majority of data holdings residing in our on-premises Data Facility at SLAC. We discuss the opportunities, risks, and technical challenges  of this approach. 
\end{abstract}

